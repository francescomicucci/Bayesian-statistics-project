\section{Problem Description}
This report examines two time series tracking the evolution of the US GDP and CPIAUCSL from 1948 to 2021. \\
The US Gross Domestic Product (GDP) represents the total market value of goods and services produced within the United States. The Consumer Price Index for All Urban Consumers (CPIAUCSL) reflects the average cost of a basket of goods and services purchased by urban consumers. Both metrics are reported quarterly and have been seasonally adjusted. In particular, we analyze their percentage changes over time. \\
The objectives of our project are as follows:
\begin{itemize}
    \item Fit each time series independently using AR, MA, ARMA, and GARCH models.
    \item Fit the two time series jointly using a VAR model.
    \item Utilize these models for in-sample and out-of-sample predictions.
    \item Compare different models using DIC and WAIC criteria.
\end{itemize}
All models were implemented using JAGS with the following specifications:
\begin{itemize}
    \item 3 Chains.
    \item Total of 10,000 Iterations.
    \item 1,000 Burn-in Iterations.
\end{itemize}
The data fed into JAGS consisted of only the first 90\% of each time series. The remaining 10\% was used to assess the out-of-sample predictions generated by the models.
Furthermore, we examined trace plots and autocorrelation plots to identify any issues and compared our findings with those obtained using publicly available libraries and functions. Detailed results are provided in the Appendix.